\documentclass[a4paper,12pt]{article}
\usepackage{graphicx}
\usepackage[left=30mm, right=30mm, top=30mm, bottom=35mm]{geometry}
\usepackage{amsmath}
\usepackage{siunitx}
\usepackage{booktabs}
\usepackage{fancyhdr}
\usepackage{url}
\pagestyle{fancy}
%-------------------------------------------------------------------------------
\lhead{\textbf{Spring 2019}}
\rhead{\textbf{CE394M Advanced Analysis in Geotechnical Engineering}}
\cfoot{\thepage}
%-------------------------------------------------------------------------------

\begin{document}
\begin{centering}
	\textbf{
		Assignment 2: Stress equilibrium, compatibility and stiffness matrix\\
		Assigned: 27th January 2020\\
		Due: 7th February 2020\\
	}
\end{centering}

\vspace{1em}
 
\begin{enumerate}
	\item Prove that the following stress-field could be a valid lower-bound solution.
	\begin{align*}
		\sigma_{xx} & = c_1 x^3 y - 2c_2 xy + c_3 y\\
		\sigma_{yy} & = c_1 x y^3 - 2c_1 x^3 y\\
		\sigma_{xy} & = -\frac{3}{2}c_1x^2y^2 + c_2 y^2 + \frac{1}{2}c_1 x^4 + c_4
	\end{align*}
	where $c_1$, $c_2$, $c_3$ and $c_4$ are constants.

	\item Determine and describe the stress-state given by the following Airy stress functions:
	\begin{enumerate}
		\item $\phi = Ay^2$ 
		\item $\phi = Bxy$
	\end{enumerate}
	Are these stresses valid for an isotropic elastic element? Note: please refer \url{https://en.wikiversity.org/wiki/Airy_stress_function} on how to determine cauchy stress components from an Airy function.

	\item Determine whether the following strain fields are possible in a two-dimensional continous body:
	
	\begin{enumerate}
		\item $\!
			\begin{aligned}[t]
				\varepsilon = %			
				\begin{bmatrix}
				\varepsilon_{xx} & \varepsilon_{xy} \\
				\varepsilon_{xy} & \varepsilon_{yy} \\
				\end{bmatrix} = 
				\begin{bmatrix}
				c_1(x^2 + y^2) & c_1 xy \\
				c_1 xy & c_2 y^2 \\
				\end{bmatrix}
			\end{aligned}$
		\item $\!
			\begin{aligned}[t]
				\varepsilon = %			
				\begin{bmatrix}
					\varepsilon_{xx} & \varepsilon_{xy} \\
					\varepsilon_{xy} & \varepsilon_{yy} \\
				\end{bmatrix} = 
				\begin{bmatrix}
				c_1(x^2 + y^3) & 3c_1 xy^2/2 \\
				3c_1 xy^2/2 & c_2 x^3 \\
				\end{bmatrix}
			\end{aligned}$		
	\end{enumerate}
	
	\item Using Optum G2 perform limit analyses of a vertical cut in clay with the following properties. Excavate to a vertical depth of $H_c$, where $H_c$ is determined using the stability charts. A preliminary stage involving development of initial stresses of a rectangular domain before excavation is required. Set this phase as the starting phase for both the lower and upper bound analyses. Compare the factor of safety from lower and upper bound solutions. Plot the displacement profiles.
	
		\begin{table}[!h]
			\centering
			\begin{tabular}{ll}
				\toprule
				\textbf{Properties}     & \textbf{Values} \\
				\midrule
				cohesion (c) & 30 kPa    \\
				friction ($\phi$)   & $0^o$   \\
				unit weight ($\gamma$)   & 20 $kN/m^3$    \\
				K0             & 0.6  \\
				\bottomrule
			\end{tabular}
		\end{table}
	
	\item Repeat the previous problem using a slope angle of $30^o$, instead of a vertical cut, and the depth to bed rock $n_d * H_c$ as  $H_c$ and $2 * H_c$.
\end{enumerate}

\end{document}

