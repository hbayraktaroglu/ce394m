\documentclass[a4paper,12pt]{article}
\usepackage{graphicx}
\usepackage[left=30mm, right=30mm, top=30mm, bottom=35mm]{geometry}
\usepackage{amsmath}
\usepackage{siunitx}
\usepackage{fancyhdr}
\usepackage{url}
\pagestyle{fancy}
%-------------------------------------------------------------------------------
\lhead{\textbf{Spring 2020}}
\rhead{\textbf{CE394M Advanced Analysis in Geotechnical Engineering}}
\cfoot{\thepage}
%-------------------------------------------------------------------------------

\begin{document}
\begin{centering}
	\textbf{
		Assignment 6: FEA, Solver, Elasticity\\
		Assigned: 6th April 2019\\
		Due: 13th April 2019 at 5 PM\\
	}
\end{centering}

\vspace{1em}

You may obtain access to Plaxis through a Virtual Desktop.  Information about gaining access can be found at \url{ http://caee.utexas.edu/students/itss/43-students/it/386-virtualdesktops}. Use UTexas VPN \url{https://wikis.utexas.edu/display/engritgpublic/Connecting+to+the+University+of+Texas+VPN} to connect to the remote desktop and other CAEE services.

 
\begin{enumerate}
	\item Two FE simulations of a 3D foundation settlement are performed using a structured mesh of tri-linear (8-noded) hexahedral elements. Assume the geometry of the mesh is cubic and all the elements are hexahedron and of the same size. Case A uses $n$ elements and Case B uses $2n$ elements, where $n$ is large.
	\begin{enumerate}
		\item How many non-zero entries would you expect in most rows of the global stiffness matrix for Case A and Case B?
		\item Compute the approximate increase in memory required to build the global stiffness matrix for Case B compared to Case A?
		\item Cases A and B are solved using an LU solver. Based on the theoretical complexity of the LU solver estimate the increase in solver time in going from Case A to Case B.
		\item If the error in $L^2$ norm of the solution is proportional to $Ch^2$, where $h$ is the length of an element edge and $C$ is a problem constant that does not depend on $h$, estimate the reduction in the error for Case B compared to Case A.
	\end{enumerate}

	\item A one-dimensional consolidation test, allowing no horizontal strain, was performed on a soil sample. At a vertical stress of 270 kPa, the vertical strain was measured as 0.2\% and the horizontal stress was measured as 115 kPa. From these results and assuming isotropic linear elasticity, develop two elastic parameters such that a complete \textbf{D} matrix can be specified for an analysis. Formulate the \textbf{D} matrix in both plane-stress and plane strain formulations.

	\item A consolidated-drained (CD) triaxial test is performed at an effective confining pressure of 100 kPa. A deviatoric stress of 20 kPa is applied. The resulting axial strain during the application of deviatoric stress is 0.08 \% and the resulting volumetric strain is 0.04\%. From these results and assuming isotropic linear elasticity, compute the radial strain and develop estimates of the shear modulus (G), bulk modulus (K), Poisson's ratio ($\nu$), and Young's modulus (E) of the soil.

	\item Familiarize yourself with the documentation of the program PLAXIS, with particular emphasis on the PLAXIS “2D 2019 – Tutorial Manual” and PLAXIS “2D 2019 – Material Models Manual”.  These can be found at \url{https://www.plaxis.com/support/manuals/plaxis-2d-manuals/}.  Please read and complete the following the Tutorial on the settlement of a Circular Footing on Sand (Sections 1.1-1.3).
	
	``Play'' with the program to ensure that you can set up, solve, and interpret a problem through the pre- and post-processor of PLAXIS.  
	
	Write a short report including the following:
	
	\begin{enumerate}
		\item Described the elements used, average mesh size, soil properties and the solver.
		
		\item Final deformed meshes for the rigid and flexible footings and the load versus displacement plot for the center of the flexible footing.

		\item Explain how the results were validated.
	\end{enumerate}
	
\end{enumerate}

\end{document}

